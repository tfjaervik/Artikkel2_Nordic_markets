\begin{table}[H]
\caption{\\ \large{\textbf{Future returns sorted on coskewness and LTD}}}
\captionsetup{font = footnotesize, justification = justified, width = \linewidth}
\caption*{Stocks have been sorted into quintiles based on their
                        estimated coskewness and LTD. For each quintile, an equal weighted return
                        for month t+1 has been calculated. This gives 1 time series per quintile of equal
                        weighted portfolio returns in month t+1. The column 
                        'Returns' report the time average of these series. *, **
                        and *** indicate statistical significance at,
                        respectively, the 10\%, 5\% and 1\% significance level.
                        
                        Note that in the presence of tied coskewness and/or LTD values, the first 
                        observation with that coskewness/LTD value is ranked lower than the 
                        following observation with the same coskewness/LTD value. This pattern
                        repeats in the presence of more than two ties. T-statistics for
                        all the hypothesis testing have been calculated using Newey
                        and West (1987) standard errors.}
\centering
\label{tab:coskewness_ltd_sorted_future_returns_equal_weighted}
\begin{tabular}[t]{llllll}
\toprule
Quintile & 1 Low Coskewness & 2 & 3 & 4 & 5 High Coskewness\\
\midrule
1 Weak LTD & 1.034\%*** & 1.163\%*** & 0.973\%*** & 0.718\%* & 0.511\%\\
2 & 1.059\%*** & 1.277\%*** & 1.016\%*** & 0.349\% & 0.577\%\\
3 & 0.89\%** & 1.089\%*** & 1.042\%*** & 0.547\% & 0.656\%\\
4 & 0.847\%** & 1.168\%*** & 1.137\%*** & 0.528\% & 0.545\%\\
5 Strong LTD & 1.11\%*** & 1.183\%*** & 1.349\%*** & 0.905\%* & 0.581\%\\
\addlinespace
Strong - Weak & 0.075\% & 0.02\% & 0.376\% & 0.186\% & 0.07\%\\
\bottomrule
\end{tabular}
\end{table}
