\begin{table}[H]
\caption{\\ \large{\textbf{Future returns sorted on coskewness and LTD}}}
\captionsetup{font = footnotesize, justification = justified, width = \linewidth}
\caption*{Stocks have been sorted into quintiles based on their
                        estimated coskewness and LTD. For each quintile, a value weighted return
                        for month t+1 has been calculated using weights from
                        month t. This gives 1 time series per quintile of value
                        weighted portfolio returns in month t+1. The column 
                        'Returns' report the time average of these series. *, **
                        and *** indicate statistical significance at,
                        respectively, the 10\%, 5\% and 1\% significance level.
                        
                        Note that in the presence of tied coskewness and/or LTD values, the first 
                        observation with that coskewness/LTD value is ranked lower than the 
                        following observation with the same coskewness/LTD value. This pattern
                        repeats in the presence of more than two ties. T-statistics for
                        all the hypothesis testing have been calculated using Newey
                        and West (1987) standard errors.}
\centering
\label{tab:coskewness_ltd_sorted_future_returns}
\begin{tabular}[t]{llllll}
\toprule
Quintile & 1 Low Coskewness & 2 & 3 & 4 & 5 High coskewness\\
\midrule
1 Weak LTD & 0.782\%** & 0.905\%*** & 0.8\%** & 0.34\% & 0.414\%\\
2 & 0.527\% & 0.835\%*** & 0.746\%** & 0.227\% & 0.508\%\\
3 & 0.755\%* & 0.661\%** & 1.063\%*** & 0.618\% & -0.01\%\\
4 & 1.128\%*** & 0.865\%*** & 1.079\%*** & 0.355\% & 0.514\%\\
5 Strong LTD & 1.003\%*** & 0.961\%** & 1.025\%** & 0.411\% & 0.333\%\\
\addlinespace
Strong - Weak & 0.221\% & 0.055\% & 0.225\% & 0.072\% & -0.081\%\\
\bottomrule
\end{tabular}
\end{table}
